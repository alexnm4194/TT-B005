\section{¿Qué es una prueba proyectiva?}

Las técnicas proyectivas son unos instrumentos considerados como especialmente sensibles para revelar aspectos inconscientes de la conducta ya que permiten provocar una amplia variedad de respuestas subjetivas, son altamente multidimensionales y evocan respuestas y datos del sujeto, inusualmente ricos con un mínimo conocimiento del objetivo del test, por parte de éste (Lindzey 1961).

Frank 1939, Bellack 1965 y ABT 1965, entre otros, han relacionado lo “proyectivo” del término con una referencia a las fuentes de comprensión e interpretación del enfoque y corriente teóricas del método psicoanalítico.
El sustrato teórico de estas técnicas es pues, el de las teorías dinámicas de la personalidad. El énfasis principal está en que proporciona una visión de la personalidad del individuo sincrática y una aceptación de un sustrato inconsciente en el que residen impulsos, tendencias, conflictos, necesidades,etc.., todas ellas inferidas del comportamiento de los individuos humanos(Fernandez Ballesteros 1981).

Existen unos principios de orientación teórica en la que se basan las Técnicas Proyectivas Gráficas:
\\
\begin{enumerate}[1)]
\item En el hombre existe una tendencia a ver el mundo de manera antropomórfica(a través de su propia imagen) y eso facilita los aspectos proyectivos implicados en los dibujos de una casa, árbol, animal y otros.
\item La esencia de la visión antropomórfica del medio es el mecanismo de proyección(por el cual uno atribuye las propias cualidades, sentimientos, actitudes y esfuerzos a objetos del entorno: personas, cosas u organismos).
\item Las distorsiones forman parte del proceso de proyección siempre que:
\begin{enumerate}[a)]
\item la proyección tenga una función defensiva
\item se invistan a los datos tangenciales, parciales o superficiales de los objetos con significados de la propia vida del sujeto, que no correspondan a la imagen real o total del objeto o al objeto cuya presencia el sujeto niega en sí mismo \cite{pruebaproyectiva}
\end{enumerate}
\end{enumerate}