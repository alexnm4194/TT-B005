\section{Expediente psicológico}

En la psicología clínica se usa un expediente clínico. Es el documento en el que se encuentra toda la historia de los pacientes, desde
que toman su primera sesión hasta que se dan de alta o bien son dados de baja; éste documento ayuda a
conocer y administrar los recursos con los que contamos, además de permitir una vista integral del trabajo, las
herramientas y las técnicas que se utilizan para resolver las necesidades de nuestros usuarios.
\newline
\newline
El expediente clínico se compone de:

\begin{itemize}
\item Hoja de datos generales. Es en la que el usuario comparte sus datos, los cuáles están protegidos y se
utilizan sólo con fines estadísticos.
\item Primera entrevista. En esta se delimitan los motivos de consulta tanto implícitos como explícitos del
usuario de nuestros servicios, así como un panorama general del inicio del problema, las soluciones
que se han presentado para resolverlo y el esquema biopsicosocial de la persona.
\item Historia Clínica. Nos da un amplio panorama de la historia del crecimiento del paciente, nos permite
conocer estilos de crianza, ideas, características familiares, sistemas de pensamiento, etc.
\item Pruebas Psicométricas y sus resultados. Son las hojas de respuesta, los resultados y la interpretación
de dichos resultado.
\item Exámenes físicos o diagnóstico médico. Aquí se incluyen los resultados de análisis físicos o
diagnósticos médicos que pudieran tener
\item Diagnóstico y Diagnóstico diferencial. Se presentan las bases para llevar a cabo un diagnóstico
basados en la interpretación de las pruebas anteriormente aplicadas, así como de los datos que se
obtuvieron durante las sesiones iniciales.
\item Plan de trabajo, es donde se especifican los objetivos a cubrir, además de las técnicas y estrategias que
se tomaran en cuenta para lograrlo
\item Notas de evolución. Son en las que se detallan las sesiones y el avance de cada una de ellas, así como
el trabajo realizado por el paciente.
\item Anexos de actividades realizadas por el paciente. las tareas entregables se anexan al expediente
clínico.
\item Hoja de baja o de alta con los motivos.
\item Contrato terapéutico, es el documento en el que se detallan las condiciones, así como los derechos y
obligaciones a las que están sujetos tanto el psicoterapeuta como el usuario de los servicios. \cite{expedientepsicologico}
\end{itemize}