\section{Tecnologías}
\subsection{iOS}
iOS es un sistema operativo móvil de la multinacional Apple Inc. Originalmente desarrollado para el iPhone (iPhone OS), después se ha usado en dispositivos como el iPod touch y el iPad. No permite la instalación de iOS en hardware de terceros.

Los elementos de control consisten de deslizadores, interruptores y botones. La respuesta a las órdenes del usuario es inmediata y provee una interfaz fluida. La interacción con el sistema operativo incluye gestos como deslices, toques, pellizcos, los cuales tienen definiciones diferentes dependiendo del contexto de la interfaz. Se utilizan acelerómetros internos para hacer que algunas aplicaciones respondan a sacudir el dispositivo o rotarlo en tres dimensiones 

iOS se deriva de macOS, que a su vez está basado en Darwin BSD, y por lo tanto es un sistema operativo Tipo Unix.

iOS cuenta con cuatro capas de abstracción: \cite{ios}
\begin{itemize}
    \item Capa del núcleo del sistema operativo
    \item Capa de ``Servicios principales''
    \item Capa de ``Medios''
    \item Capa de ``Cocoa Touch''
\end{itemize}

\subsection{Swift}
Swift es un lenguaje de programación poderoso e intuitivo creado por Apple para desarrollar apps de iOS, Mac, Apple TV y Apple Watch. Está diseñado para brindar a los desarrolladores más libertad que nunca. Y como es fácil de usar y de código abierto, es ideal para que cualquier persona con una idea pueda hacerla realidad.

Swift es gratis y de código abierto, y está disponible para desarrolladores, educadores y estudiantes bajo la licencia de código abierto Apache 2.0. \cite{swift}
\newline

\subsection{Xcode}
Xcode es un entorno de desarrollo integrado (IDE, en sus siglas en inglés) para macOS que contiene un conjunto de herramientas creadas por Apple destinadas al desarrollo de software para macOS, iOS, watchOS y tvOS. Su primera versión tiene origen en el año 2003 y actualmente su versión número 8 se encuentra disponible de manera gratuita en el Mac App Store o mediante descarga directa desde la página para desarrolladores de Apple.

Xcode incluye la colección de compiladores del proyecto GNU (GCC), y puede compilar código C, C++, Swift, Objective-C, Objective-C++, Java y AppleScript \cite{xcode}

\newpage
\subsection{Firebase}
Firebase es la nueva y mejorada plataforma de desarrollo móvil en la nube de Google. Se trata de una plataforma disponible para diferentes plataformas (Android, iOS, web)

Sus características fundamentales están divididas en varios grupos, las cuales podemos agrupar en: \cite{firebase}

\begin{itemize}
    
\item Analíticas: Provee una solución gratuita para tener todo tipo de medidas (hasta 500 tipos de eventos), para gestionarlo todo desde un único panel.
\item Desarrollo: Permite construir mejores apps, permitiendo delegar determinadas operaciones en Firebase, para poder ahorrar tiempo, evitar bugs y obtener un aceptable nivel de calidad. Entre sus características destacan el almacenamiento, testeo, configuración remota, mensajería en la nube o autenticación, entre otras.
\item Crecimiento: Permite gestionar los usuarios de las aplicaciones, pudiendo además captar nuevos. Para ello dispondremos de funcionalidades como las de invitaciones, indexación o notificaciones.
\item Monetización: Permite ganar dinero gracias a AdMob.

\end{itemize}
%%\subsection{SQLite}
