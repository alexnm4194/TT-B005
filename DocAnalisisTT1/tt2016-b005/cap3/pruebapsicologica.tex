\section{¿Qué es una prueba psicológica?}

Definiremos ``prueba psicológica'' como el proceso de medir variables relacionadas con la psicología por medio de dispositivos o procedimientos diseñados para obtener una muestra de comportamiento (Cohen y Swerdlik, 2001). \cite{pruebapsicologica1}

La Psicología reconoce en la Psicometría esa rama que se ocupa de las cuestiones relacionadas con la medición, y si bien es cierto que las ciencias atraviesan una época de crisis de paradigmas y en especial las ciencias sociales y conductuales, aún así podemos encontrar contenidos tradicionales en la Psicometría que son punto de acuerdo entre la mayoría de los autores e investigadores de la Psicología. Se podrían sintetizar en tres ejes: 
\\\\
\begin{enumerate}[A)]
\item Los procesos operacionales de medición en Psicología asociados a las escalas de medida: el objetivo de la Psicometría será hallar la mejor manera de observar, clasificar y transformar categorías manifiestas en escalas ``cuantitativas'' partiendo de la aceptación del isomorfismo entre propiedades atribuidas a las categorías psicológicas y las propiedades atribuidas a los números que las representan (Stevens, 1951)
\item Confiabilidad o precisión de los instrumentos de medida en Psicología: es uno de los tres problemas de medida asociados a las escalas de medida que merecen atención ya que si una prueba psicométrica no es confiable en su medición, su inconsistencia repercutirá negativamente no solo en la validez del instrumento sino en todos los procesos relacionales que se incluyan.
\item Validez de una prueba: es la propiedad fundamental en tanto permite decir de una prueba que mide lo que pretende medir y es un “valor social sobresaliente que asume una función tanto científica como política (Messick, 1995) \cite{pruebapsicologica2}
\end{enumerate}