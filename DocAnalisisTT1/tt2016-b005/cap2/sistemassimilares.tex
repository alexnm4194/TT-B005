\section{Sistemas comerciales}

En la siguiente tabla (Tabla \ref{tab:sistemComercial} Sistemas comerciales) se muestran algunos sistemas con características similares, así como el tipo de licencia que estos manejan

\begin{table}[htbp]
    \centering
    \begin{tabular}{ |m{10em}|m{4em}|m{4em}|m{4.5em}|m{3em}| }
    \hline
    Nombre & Internet & Gestor de pacientes & Aplicación de pruebas & Licencia
    \\
    \hline
    \textbf{Control de pacientes (Beta)} & Obligatorio & X & N/A & Gratis
    \\
    \hline
    \textbf{Insight} & Opcional & X & N/A & pago anual
    \\
    \hline
    \textbf{jclic} & Opcional & N/A & X & Gratis
    \\
    \hline
     \textbf{Sistema para aplicación de la prueba proyectiva HTP en apoyo a psicólogos} & Opcional & X & X & -
     \\
     \hline
    \end{tabular}
    \caption{Sistemas comerciales. Fuente: Creación propia}
    \label{tab:sistemComercial}
\end{table}

\newpage
\subsection{Control de pacientes (Beta)}
Control de pacientes (Beta) es un software web gratuito para el control total de la gestión de la historia médica de los pacientes.

Al ser una plataforma web no requiere de instalación y puede utilizarse en cualquier dispositivo.

Esta mas enfocada al sector médico al permitir la creación digital de recetas médicas y configuración de tratamientos.

Cuenta con un calendario para programas citas a futuro, así como la gestión de la consulta médica \cite{ControlPacientes}

\subsection{Insight}
Insight se encuentra dividido en 5 secciones principales:

Tareas administrativas: permite la creación de empresas, gestión de los usuarios de la aplicación, copias de seguridad, importar datos externos, etc.

Gestión de pacientes: permite introducir datos de los pacientes, historial farmacológico, minusvalías, diagnósticos, datos del tratamiento, sesiones aplicadas al paciente, informes(documentos anexos), plantillas de informes, gestión de pagos, facturas, bonos, etc. 

Contabilidad: permite llevar la contabilidad transparentemente para el usuario. Realiza una generación automática de los asientos contables asociados a las facturas, generación de cuentas contables automáticas asociadas a clientes y proveedores. 

Herramientas: dispone de agenda conectada con un sistema de avisos con mensajes SMS y Email. La agenda es accesible a través de dispositivos móviles. Además la agenda tiene la posibilidad de ofrecer un servicio de Cita en linea.

Base de datos en la clínica/Base de datos en la nube: permite ubicar la base de datos en su propia clínica para que se conecten los equipos de su red cableada, wifi, o incluso desde fuera de la clínica. También es posible ubicar la base de datos en la nube; de manera que podrá conectarse desde cualquier ubicación, cualquier día y hora. \cite{insight}

\vspace{1cm}

\subsection{jclic}
JClic está formado por un conjunto de aplicaciones informáticas que sirven para realizar diversos tipos de actividades educativas: rompecabezas, asociaciones, ejercicios de texto, palabras cruzadas, etc.

Las actividades no se acostumbran a presentar solas, sino empaquetadas en proyectos. Un proyecto está formado por un conjunto de actividades y una o más secuencias, que indican el orden en qué se han de mostrar.

JClic está desarrollado en la plataforma Java, es un proyecto de código abierto y funciona en diversos entornos y sistemas operativos.\cite{jclic}
