\section{Contexto}

El campo de la psicología es uno de los menos considerados para la implementación de sistemas, esto debido a la naturaleza de la psicología que es subjetiva y depende de numerosas variables que no siempre pueden ser computables ya que no es una ciencia exacta, esto provoca que no se puedan desarrollar modelos computacionales que entreguen resultados precisos, ya que no solo dependen del sujeto de estudio, involucran la formación, experiencia y criterio del analista.

Sin embargo gracias al avance tecnológico es posible empezar a crear herramientas que ayuden a los psicólogos e ir incorporando diversas tecnologías con su área de estudio de forma gradual para observar el comportamiento de estas dos ramas para de esta forma sea un apoyo y no se pierda la esencia de la interacción humana.

En la actualidad existen sistemas que ayudan a los psicólogos en la gestión de citas, control de pacientes, generación de reportes semi-automatizados y la aplicación de pruebas tipo cuestionario donde cada respuesta tiene un valor cuantificable y constante.

El uso de herramientas psicológicas como pruebas proyectivas gráficas, permite al área de sistemas computacionales tomar los elementos cuantitativos y cualitativos que pueden ser computados como lo es el trazo y/o la posición de un dibujo elaborado sobre dispositivos táctiles, y de esta manera poder diseñar prototipos de modelos computacionales que ayuden en su trabajo a los psicólogos, de esta manera impulsar y promover la integración de tecnologías en el área de la psicología.