%Crear Nota
\begin{UseCase}{CU7}{Crear Cuestionario}{
    El psicólogo podrá crear cuestionarios de forma dinámica en los cueales podrá poner preguntas abiertas, cerradas y de opción multiple para apliarle a sus diferentes pacientes.
	}
	\UCitem{Versión}{1.0 - 11/04/2017}
	\UCitem{Autor}{Alejandro Noriega Montalban}
	\UCitem{Prioridad}{Alta}
	\UCitem{Actor}{Psicólogo}
	\UCitem{Propósito}{Crear cuestionarios.}
	\UCitem{Entradas}{Formulario de cuestionarios}
	\UCitem{Salidas}{Cuestionario de aplicación}
	\UCitem{Precondiciones}{
	}
	\UCitem{Postcondiciones}{
	}
	\UCitem{Reglas del negocio}{
	    \begin{itemize}
	        \item \BRref{RN25}{ Formato Nombres}
            \item \BRref{RN27}{ Tipos de preguntas}
            \item \BRref{RN28}{ Ignorar campos}
	}	
	\UCitem{Mensajes}{
    	\begin{itemize}
    	    \item \MSGref{MSG1}{Operación exitosa}
    	    \item \MSGref{MSG2}{Operación fallida}
    	    \item \MSGref{MSG4}{Confirmación}
    	    \item \MSGref{MSG6}{Formato incorrecto registro}
    	\end{itemize}
\end{UseCase}
%.... Pendiente reglas del neogocio

% ------------ Trayectoria principal, poner referencias a pantallas del sistema, reglas entre otras cosas que sean necesarias. ------------
\begin{UCtrayectoria}{Principal}
    \UCpaso Muestra la pantalla \IUref{UIPPrincipal}{Pantalla principal del sistema}.
    \UCpaso[\UCactor] Oprime el botón \IUbutton{Cuestionarios}.
    \UCpaso Muestra \IUref{UIC}{Pantalla Cuestionarios}.
    \UCpaso[\UCactor] Oprime el botón \IUbutton{Crear cuestionario}.
    \UCpaso Muestra \IUref{UIN}{Pantalla Nuevo Cuestionario}.
    \UCpaso[\UCactor] Se introduce el nombre en el campo.
    \UCpaso[\UCactor] Se introduce el texto en el campo pregunta 1. \Trayref{C}\Trayref{C2} \label{CU7Regresa2}
    \UCpaso[\UCactor] Se introduce el texto en el campo pregunta 2. \Trayref{C}\Trayref{C2}
    \UCpaso[\UCactor] Oprime el botón \IUbutton{Guardar}. \label{CU7Regresa} \Trayref{D} \Trayref{R}
    \UCpaso Verifica el campo de nombre. \Trayref{A}
    	\\\BRref{RN25}{ Formato Nombres}
    \UCpaso Ignora los campos vacios.
        \\\BRref{RN28}{ Ignorar campos}
    \UCpaso Muestra el Mensaje \textbf{\MSGref{MSG4}{Confirmación}}.
    \UCpaso[\UCactor] Oprime el botón \IUbutton{Aceptar}.
    \UCpaso Se guarda el cuestionario. \Trayref{B}
    \UCpaso Muestra el Mensaje \textbf{\MSGref{MSG1}{Operación exitosa}}.
    \UCpaso[\UCactor] Oprime el botón \IUbutton{Aceptar}.
    \UCpaso Muestra la pantalla \IUref{UIC}{Pantalla Cuestionarios}.
    %\UCpaso Continúa en el \UCref{CU23}.
\end{UCtrayectoria}

% ---------------- Trayectorias alternativas -------------- Colocar los mensajes  {\bf MSG1-}

\begin{UCtrayectoriaA}{A}{El campo no tiene el formato correcto}
    \UCpaso Muestra el Mensaje \textbf{\MSGref {MSG6}{Formato incorrecto registro}}.
    \UCpaso Muestra \IUref{UIN}{Pantalla Nuevo Cuestionario} con el campo en color rojo, conservando los datos y el texto ingresado.
	\UCpaso Continua en el paso \ref{CU7Regresa} del \UCref{CU7}
\end{UCtrayectoriaA}

\begin{UCtrayectoriaA}{B}{No se pudo crear el cuestionario}
    \UCpaso Muestra el Mensaje \textbf{\MSGref{MSG2}{Operación fallida}}.
    \UCpaso[\UCactor] Oprime el botón \IUbutton{Aceptar}.
    \UCpaso Muestra \IUref{UIN}{Pantalla Nuevo Cuestionario} conservando los datos ingresados.
    \UCpaso Continua en el paso \ref{CU7Regresa} del \UCref{CU7}
\end{UCtrayectoriaA}

\begin{UCtrayectoriaA}{C}{Pregunta tipo cerrada}
    \UCpaso[\UCactor] Selecciona el check \IUbutton{carrada} para la pregunta.
    \UCpaso Se habilitan los campos para ingresar respuestas.
        \\\BRref{RN27}{ Tipos de preguntas}
    \UCpaso Se ingresa el texto en los campos de respuestas.
    \UCpaso Continua en el paso \ref{CU7Regresa2} del \UCref{CU7}
\end{UCtrayectoriaA}

\begin{UCtrayectoriaA}{C2}{Pregunta tipo Multiple}
    \UCpaso[\UCactor] Selecciona el check \IUbutton{Multiple} para la pregunta.
    \UCpaso Se habilitan los campos para ingresar respuestas.
        \\\BRref{RN27}{ Tipos de preguntas}
    \UCpaso Se ingresa el texto en los campos de respuestas.
    \UCpaso Continua en el paso \ref{CU7Regresa2} del \UCref{CU7}
\end{UCtrayectoriaA}

\begin{UCtrayectoriaA}{D}{Agregar preguntas}
    \UCpaso[\UCactor] Oprime el botón \IUbutton{Nueva pregunta}.
    \UCpaso Muestra \IUref{UIN}{Pantalla Nuevo Cuestionario} con el campo del nombre conservando los datos.
    \UCpaso Continua en el paso \ref{CU7Regresa2} del \UCref{CU7}
\end{UCtrayectoriaA}

\begin{UCtrayectoriaA}{R}{Se cancela el registro}
	\UCpaso[\UCactor] Oprime el botón \IUbutton{Cancelar}.
	\UCpaso Muestra el Mensaje \textbf{\MSGref{MSG4}{Confirmación}}.
    \UCpaso[\UCactor] Oprime el botón \IUbutton{Aceptar}. \Trayref{R2}
	\UCpaso Muestra \IUref{UIC}{Pantalla Cuestionarios}.
	\UCpaso Termina el caso de uso.
\end{UCtrayectoriaA}

\begin{UCtrayectoriaA}{R2}{Se cancela la operación cancelar}
    \UCpaso[\UCactor] Oprime el botón \IUbutton{Cancelar} del mensaje
        \\\textbf{\MSGref{MSG4}{Confirmación}}.
    \UCpaso Muestra \IUref{UIN}{Pantalla Nuevo Cuestionario} conservando los datos ingresados.
    \UCpaso Continua en el paso \ref{CU7Regresa} del \UCref{CU7}
\end{UCtrayectoriaA}
%-------------------------------------- TERMINA descripción del caso de uso.