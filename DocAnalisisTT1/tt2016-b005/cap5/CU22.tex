%Editar Nota
\begin{UseCase}{CU22}{Editar Nota}{
    El psicólogo podrá modificar o eliminar las notas creadas.
	}
	\UCitem{Versión}{1.0 - 10/04/2017}
	\UCitem{Autor}{Alejandro Noriega Montalban}
	\UCitem{Prioridad}{Baja}
	\UCitem{Actor}{Psicólogo}
	\UCitem{Propósito}{Modificar una nota.}
	\UCitem{Entradas}{Texto}
	\UCitem{Salidas}{}
	\UCitem{Precondiciones}{
        \begin{itemize}
            \item La nota debe existir en el sistema.
        \end{itemize}
	}
	\UCitem{Postcondiciones}{
        \begin{itemize}
            \item La nota quedará actualizada en el sistema.
        \end{itemize}
	}
	\UCitem{Reglas del negocio}{
	    \begin{itemize}
	        \item \BRref{RN25}{ Formato Nombres}
	        \item \BRref{RN26}{ Notas del paciente}
	    \end{itemize}
	}	
	\UCitem{Mensajes}{
    	\begin{itemize}
    	    \item \MSGref{MSG1}{Operación exitosa}
    	    \item \MSGref{MSG2}{Operación fallida}
    	    \item \MSGref{MSG4}{Confirmación}
    	    \item \MSGref{MSG6}{Formato incorrecto registro} 
    	\end{itemize}
    }
\end{UseCase}
%.... Pendiente reglas del neogocio

% ------------ Trayectoria principal, poner referencias a pantallas del sistema, reglas entre otras cosas que sean necesarias. ------------
\begin{UCtrayectoria}{Principal}
    \UCpaso Muestra la pantalla \IUref{UIP}{Pantalla Información paciente}.
    \UCpaso[\UCactor] Oprime el botón \IUbutton{Notas}.
    \UCpaso Muestra \IUref{UIN}{Pantalla Notas}.
    \UCpaso[\UCactor] Selecciona una nota.
    \UCpaso Despliega en pantalla la nota vía
        \\\IUref{UIEN}{Pantalla Espacio nota}.
    \UCpaso[\UCactor] Oprime el botón \IUbutton{Editar}.
    \UCpaso[\UCactor] Se introduce el nuevo nombre en el campo.
    \UCpaso[\UCactor] Se edita el texto existente.
    \UCpaso[\UCactor] Oprime el botón \IUbutton{Guardar}. \label{CU21Regresa} \Trayref{R} \Trayref{C}
    \UCpaso Verifica el campo de nombre. \Trayref{A}
    	\\\BRref{RN25}{ Formato Nombres}
    \UCpaso Muestra el Mensaje \textbf{\MSGref{MSG4}{Confirmación}}.
    \UCpaso[\UCactor] Oprime el botón \IUbutton{Aceptar}.
    \UCpaso Se guarda la nota. \Trayref{B}
    	\\\BRref{RN26}{ Notas del paciente}
    \UCpaso Muestra el Mensaje \textbf{\MSGref{MSG1}{Operación exitosa}}.
    \UCpaso[\UCactor] Oprime el botón \IUbutton{Aceptar}.
    \UCpaso Muestra la pantalla \IUref{UIP}{Pantalla Información paciente}.
    \UCpaso Continúa en el \UCref{CU23}.
\end{UCtrayectoria}

% ---------------- Trayectorias alternativas -------------- Colocar los mensajes  {\bf MSG1-}

\begin{UCtrayectoriaA}{A}{El campo no tiene el formato correcto}
    \UCpaso Muestra el Mensaje \textbf{\MSGref {MSG6}{Formato incorrecto registro}}.
    \UCpaso Muestra \IUref{UIEN}{Pantalla Espacio nota} con el campo en color rojo, conservando los datos y el texto ingresados.
	\UCpaso Continua en el paso \ref{CU21Regresa} del \UCref{CU21}
\end{UCtrayectoriaA}

\begin{UCtrayectoriaA}{B}{No se pudo crear la nota}
    \UCpaso Muestra el Mensaje \textbf{\MSGref{MSG2}{Operación fallida}}.
    \UCpaso[\UCactor] Oprime el botón \IUbutton{Aceptar}.
    \UCpaso Muestra \IUref{UIEN}{Pantalla Espacio nota} conservando los datos ingresados.
    \UCpaso Continua en el paso \ref{CU21Regresa} del \UCref{CU21}
\end{UCtrayectoriaA}

\begin{UCtrayectoriaA}{C}{Se desea eliminar la nota}
    \UCpaso[\UCactor] Oprime el botón \IUbutton{Eliminar}.
    \UCpaso Muestra el Mensaje \textbf{\MSGref{MSG4}{Confirmación}}.
    \UCpaso[\UCactor] Oprime el botón \IUbutton{Aceptar}. \Trayref{R2}
    \UCpaso Muestra \IUref{UIN}{Pantalla Notas}.
    \UCpaso Termina el caso de uso.
\end{UCtrayectoriaA}

\begin{UCtrayectoriaA}{R}{Se cancela el registro}
	\UCpaso[\UCactor] Oprime el botón \IUbutton{Cancelar}.
	\UCpaso Muestra el Mensaje \textbf{\MSGref{MSG4}{Confirmación}}.
    \UCpaso[\UCactor] Oprime el botón \IUbutton{Aceptar}. \Trayref{R2}
	\UCpaso Muestra \IUref{UIP}{Pantalla Información paciente}.
	\UCpaso Termina el caso de uso.
\end{UCtrayectoriaA}

\begin{UCtrayectoriaA}{R2}{Se cancela la operación previa}
    \UCpaso[\UCactor] Oprime el botón \IUbutton{Cancelar} del mensaje
        \\\textbf{\MSGref{MSG4}{Confirmación}}.
    \UCpaso Muestra \IUref{UIEN}{Pantalla Espacio nota} conservando los datos ingresados.
    \UCpaso Continua en el paso \ref{CU21Regresa} del \UCref{CU21}
\end{UCtrayectoriaA}
%-------------------------------------- TERMINA descripción del caso de uso.