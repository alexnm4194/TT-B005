%-------------------- COMIENZA descripción del caso de uso.

\begin{UseCase}{CU3}{Registrar paciente}{
		El psicólogo proporcionará los datos personales solicitados por el sistema, los cuales están descritos en el modelo conceptual del negocio.
	}
	\UCitem{Versión}{1.3 - 09/04/2017}
	\UCitem{Autor}{Alejandro Noriega Montalban}
	\UCitem{Prioridad}{Alta}
	\UCitem{Actor}{Psicólogo}
	\UCitem{Propósito}{Registrar un paciente.}
	\UCitem{Entradas}{Formulario de registro de pacientes}
	\UCitem{Salidas}{}
	\UCitem{Precondiciones}{
        \begin{itemize}
            \item El paciente no debe estar registrado en el sistema.
        \end{itemize}
	}
	\UCitem{Postcondiciones}{
	    \begin{itemize}
            \item Se creará un expediente nuevo para el paciente en el sistema.
        \end{itemize}
	}
	\UCitem{Reglas del negocio}{
	    \begin{itemize}
	        \item \BRref{RN1}{Datos requeridos}
	        \item \BRref{RN2}{Datos correctos}
	        \item \BRref{RN3}{Unicidad de elementos}
	        \item \BRref{RN4}{Datos obtenidos paciente}
	        \item \BRref{RN9}{Enteros Positivos}
	    \end{itemize}
	}	
	\UCitem{Mensajes}{
    	\begin{itemize}	
    	    \item \MSGref {MSG1}{Operación exitosa}
    	    \item \MSGref {MSG2}{Operación fallida}
    	    \item \MSGref {MSG4}{Confirmación}
    	    \item \MSGref {MSG5}{Registro repetido}
    	    \item \MSGref {MSG6}{Formato incorrecto registro}
    		\item \MSGref {MSG8}{Dato requerido}
    	\end{itemize}
	}
\end{UseCase}


% ------------ Trayectoria principal, poner referencias a pantallas del sistema, reglas entre otras cosas que sean necesarias. ------------
\begin{UCtrayectoria}{Principal}
    \UCpaso Muestra la pantalla \\\IUref{UIPPrincipal}{Pantalla principal del sistema}.
    \UCpaso[\UCactor] Oprime el botón \IUbutton{Nuevo Paciente}.
    \UCpaso Muestra \IUref{UIEP}{Pantalla Nuevo paciente}.
	\UCpaso[\UCactor] Llena los campos con la información solicitada por el sistema.\label{CU3Regresa}
	\\\BRref{RN4}{Datos obtenidos paciente}
	\UCpaso[\UCactor] Oprime el botón \IUbutton{Registrar}.\Trayref{R}
	\UCpaso Verifica que los campos requeridos no estén vacíos.\Trayref{A}
	    \\\BRref{RN1}{Datos requeridos}
	\UCpaso Verifica que que todos campos tengan un formato valido.\Trayref{B}
	    \\\BRref{RN2}{Datos correctos}
	    \\\BRref{RN9}{Enteros Positivos}
	\UCpaso Verifica que la boleta no exista en el sistema.\Trayref{C}\Trayref{E}
	    \\\BRref{RN3}{Unicidad de elementos}
	\UCpaso Se crea el expediente del paciente.\Trayref{D}
	\UCpaso Muestra el Mensaje \textbf{\MSGref{MSG1}{Operación exitosa}}.
    \UCpaso[\UCactor] Oprime el botón \IUbutton{Aceptar}.
	\UCpaso Muestra la pantalla \\\IUref{UIPPrincipal}{Pantalla principal del sistema}.
\end{UCtrayectoria}

% ---------------- Trayectorias alternativas -------------- Colocar los mensajes  {\bf MSG1-}

\begin{UCtrayectoriaA}{A}{EL campo se encuentra vacío}
	\UCpaso Muestra el Mensaje \textbf{\MSGref{MSG8}{Dato requerido}}.
    \UCpaso Muestra \IUref{UIEP}{Pantalla Nuevo paciente} con el campo en color azul y conservando los demás datos ingresados.
	\UCpaso Continua en el paso \ref{CU3Regresa} del \UCref{CU3}
\end{UCtrayectoriaA}
		
\begin{UCtrayectoriaA}{B}{El campo no tiene el formato correcto}
    \UCpaso Muestra el Mensaje \textbf{\MSGref {MSG6}{Formato incorrecto registro}}.
    \UCpaso Muestra \IUref{UIEP}{Pantalla Nuevo paciente} con el campo en color rojo y conservando los demás datos ingresados.
	\UCpaso Continua en el paso \ref{CU3Regresa} del \UCref{CU3}
\end{UCtrayectoriaA}

\begin{UCtrayectoriaA}{C}{El registro ya existe en el sistema}
    \UCpaso Muestra el Mensaje \textbf{\MSGref {MSG5}{Registro repetido}}.
    \UCpaso Se limpia el campo.
    \UCpaso Muestra \IUref{UIEP}{Pantalla Nuevo paciente} conservando los demás datos ingresados.
	\UCpaso Continua en el paso \ref{CU3Regresa} del \UCref{CU3}
\end{UCtrayectoriaA}

\begin{UCtrayectoriaA}{D}{No se pudo crear el expediente}
    \UCpaso Muestra el Mensaje \textbf{\MSGref{MSG2}{Operación fallida}}.
    \UCpaso[\UCactor] Oprime el botón \IUbutton{Aceptar}.
    \UCpaso Muestra \IUref{UIEP}{Pantalla Nuevo paciente} conservando los datos ingresados.
    \UCpaso Continua en el paso \ref{CU3Regresa} del \UCref{CU3}
\end{UCtrayectoriaA}

\begin{UCtrayectoriaA}{E}{No hay conexión a internet}
    \UCpaso Muestra el Mensaje \textbf{\MSGref{MSG2}{Operación fallida}}.
    \UCpaso[\UCactor] Oprime el botón \IUbutton{Aceptar}.
    \UCpaso Muestra \IUref{UIEP}{Pantalla Nuevo paciente} conservando los datos ingresados.
    \UCpaso Continua en el paso \ref{CU3Regresa} del \UCref{CU3}
\end{UCtrayectoriaA}

\begin{UCtrayectoriaA}{R}{Se cancela el registro}
	\UCpaso[\UCactor] Oprime el botón \IUbutton{Cancelar}.
	\UCpaso Muestra el Mensaje \textbf{\MSGref{MSG4}{Confirmación}}.
    \UCpaso[\UCactor] Oprime el botón \IUbutton{Aceptar}. \Trayref{R2}
	\UCpaso Muestra \IUref{UIPPrincipal}{Pantalla principal del sistema}.
	\UCpaso Termina el caso de uso.
\end{UCtrayectoriaA}

\begin{UCtrayectoriaA}{R2}{Se cancela la operación cancelar}
    \UCpaso[\UCactor] Oprime el botón \IUbutton{Cancelar} del mensaje
        \\\textbf{\MSGref{MSG4}{Confirmación}}.
    \UCpaso Muestra \IUref{UIEP}{Pantalla Nuevo paciente} conservando los datos ingresados.
    \UCpaso Continua en el paso \ref{CU3Regresa} del \UCref{CU3}
\end{UCtrayectoriaA}
		
%-------------------------------------- TERMINA descripción del caso de uso.