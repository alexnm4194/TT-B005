%Buscar paciente
\begin{UseCase}{CU6}{Buscar paciente}{
    El psicólogo tendrá como herramienta un buscador para poder filtrar a todos sus pacientes y de esta forma facilitarle el encontrar un paciente en especifico en la lista de pacientes.
	}
	\UCitem{Versión}{1.2 - 09/04/2017}
	\UCitem{Autor}{Alejandro Noriega Montalban}
	\UCitem{Prioridad}{Baja}
	\UCitem{Actor}{Psicólogo}
	\UCitem{Propósito}{El psicólogo encuentre más fácilmente a un paciente.}
	\UCitem{Entradas}{Campo de búsqueda pacientes}
	\UCitem{Salidas}{}
	\UCitem{Precondiciones}{
        \begin{itemize}
            \item El expediente del paciente debe existir en el sistema.
        \end{itemize}
	}
	\UCitem{Postcondiciones}{
	}
	\UCitem{Reglas del negocio}{
	    \begin{itemize}
	        \item \BRref{RN9}{Enteros Positivos}
	        \item \BRref{RN20}{Búsqueda}
	        \item \BRref{RN21}{Formato de Búsqueda}
	        \item \BRref{RN22}{Búsqueda paciente}
	    \end{itemize}
	}	
	\UCitem{Mensajes}{
    	\begin{itemize}
    	    \item \MSGref {MSG3}{No existe información}
    	    \item \MSGref {MSG14}{Formato de búsqueda}
    	\end{itemize}
	}
\end{UseCase}
%.... Pendiente reglas del neogocio

% ------------ Trayectoria principal, poner referencias a pantallas del sistema, reglas entre otras cosas que sean necesarias. ------------
\begin{UCtrayectoria}{Principal}
    \UCpaso Muestra la pantalla \\\IUref{UIPPrincipal}{Pantalla principal del sistema}.
    \UCpaso[\UCactor] Oprime el botón \IUbutton{Ver Pacientes}.
    \UCpaso Muestra \IUref{UIVP}{Pantalla Ver pacientes}. \Trayref{R}
    \UCpaso[\UCactor] Introduce un carácter en el campo de búsqueda. \label{CU6Regresa}
    \UCpaso Verifica el campo de búsqueda. \Trayref{A} \label{CU6Regresa2}
        \\\BRref{RN9}{Enteros Positivos}
	    \\\BRref{RN21}{Formato de Búsqueda}
	\UCpaso Realiza la búsqueda. \Trayref{B}
    	\\\BRref{RN22}{Búsqueda paciente}
	\UCpaso Muestra \IUref{UIVP}{Pantalla Ver pacientes} con los pacientes encontrados. \Trayref{C}
        \\\BRref{RN20}{Búsqueda}
    %\UCpaso Continúa en el paso \ref{CU6Regresa} del \UCref{CU6}.
\end{UCtrayectoria}

% ---------------- Trayectorias alternativas -------------- Colocar los mensajes  {\bf MSG1-}

\begin{UCtrayectoriaA}{A}{El campo de búsqueda no tiene formato correcto}
    \UCpaso Muestra el Mensaje \textbf{\MSGref {MSG14}{Formato de búsqueda}}.
    \UCpaso Muestra \IUref{UIVP}{Pantalla Ver pacientes} conservando el texto en el campo y la lista sin pacientes.
	\UCpaso Continua en el paso \ref{CU6Regresa} del \UCref{CU6}
\end{UCtrayectoriaA}
		
\begin{UCtrayectoriaA}{B}{La búsqueda no arrojo resultados}
    \UCpaso Muestra el Mensaje \textbf{\MSGref {MSG3}{No existe información}}.
    \UCpaso Muestra \IUref{UIVP}{Pantalla Ver pacientes} conservando el texto en el campo y la lista sin pacientes.
	\UCpaso Continua en el paso \ref{CU6Regresa} del \UCref{CU6}	
\end{UCtrayectoriaA}

\begin{UCtrayectoriaA}{C}{Se ingresa un nuevo carácter}
    \UCpaso Continua en el paso \ref{CU6Regresa2} del \UCref{CU6}
\end{UCtrayectoriaA}

\begin{UCtrayectoriaA}{R}{Se cancela la búsqueda}
	\UCpaso[\UCactor] Oprime el botón \IUbutton{Regresar}.
	\UCpaso Se limpia el campo de búsqueda.
	\UCpaso Muestra la pantalla \\\IUref{UIPPrincipal}{Pantalla principal del sistema}.
	\UCpaso Termina el caso de uso.
\end{UCtrayectoriaA}
%-------------------------------------- TERMINA descripción del caso de uso.