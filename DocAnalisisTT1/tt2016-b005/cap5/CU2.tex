%-------------------- COMIENZA descripción del caso de uso.

\begin{UseCase}{CU2}{Editar información del psicólogo}{
		El psicólogo proporcionará los datos personales solicitados por el sistema, los cuales desea modificar dentro de su cuenta.
	}
	\UCitem{Versión}{1.5 - 09/04/2017}
	\UCitem{Autor}{Alejandro Noriega Montalban}
	\UCitem{Prioridad}{Baja}
	\UCitem{Actor}{Psicólogo}
	\UCitem{Propósito}{Modificar la información del psicólogo en el sistema.}
	\UCitem{Entradas}{Formulario de editar información psicólogo}
	\UCitem{Salidas}{Información actualizada}
	\UCitem{Precondiciones}{
        \begin{itemize}
            \item El psicólogo debe estar registrado en el sistema.
            \item La cuenta debe estar verificada.
        \end{itemize}
	}
	\UCitem{Postcondiciones}{
	    \begin{itemize}
            \item La nueva información quedará almacenada en el sistema.
        \end{itemize}
	}
	\UCitem{Reglas del negocio}{
	    \begin{itemize}
	        \item \BRref{RN1}{Datos requeridos}
	        \item \BRref{RN2}{Datos correctos}
	        \item \BRref{RN5}{Datos obtenidos psicólogo}
	        \item \BRref{RN16}{Confirmación de datos}
	        \item \BRref{RN17}{Restricción de cambios}
    		\item \BRref{RN18}{Cambio en la contraseña}
	    \end{itemize}
	}	
	\UCitem{Mensajes}{
    	\begin{itemize}	
    	    \item \MSGref {MSG1}{Operación exitosa}
    	    \item \MSGref {MSG2}{Operación fallida}
    	    \item \MSGref {MSG4}{Confirmación}
    	    \item \MSGref {MSG6}{Formato incorrecto registro}
    		\item \MSGref {MSG8}{Dato requerido}
    		\item \MSGref {MSG10}{Datos no coinciden}
    	\end{itemize}
	}
\end{UseCase}

% ------------ Trayectoria principal, poner referencias a pantallas del sistema, reglas entre otras cosas que sean necesarias. ------------
\begin{UCtrayectoria}{Principal}
    \UCpaso Muestra la pantalla \\\IUref{UIPPrincipal}{Pantalla principal del sistema}.
    \UCpaso[\UCactor] Oprime el botón \IUbutton{Editar perfil}.
    \UCpaso Muestra \IUref{UIEP}{Pantalla Editar perfil} con la información que se tiene almacenada en la base de datos.
    \UCpaso Restringe los cambios en algunos campos.
        \\\BRref{RN17}{Restricción de cambios}
    \UCpaso[\UCactor] Edita los campos que se desea cambiar.\label{CU2Regresa}
    	\\\BRref{RN5}{Datos obtenidos psicólogo} 
	\UCpaso[\UCactor] Oprime el botón \IUbutton{Actualizar}.\Trayref{R} 
	\UCpaso Verifica que los campos requeridos no estén vacíos.\Trayref{A}
	    \\\BRref{RN1}{Datos requeridos}
	\UCpaso Verifica que que todos campos tengan un formato valido.\Trayref{B}
	    \\\BRref{RN2}{Datos correctos}
	\UCpaso Verifica que el campo de contraseña actual este vacío.\Trayref{C}
	\UCpaso Se actualiza la información de la cuenta.\Trayref{E} \Trayref{F} \label{CU2Regresa2}
	\UCpaso Muestra el Mensaje \textbf{\MSGref{MSG1}{Operación exitosa}}.
    \UCpaso[\UCactor] Oprime el botón \IUbutton{Aceptar}.
	\UCpaso Muestra la pantalla \\\IUref{UIPPrincipal}{Pantalla principal del sistema}.
\end{UCtrayectoria}

% ---------------- Trayectorias alternativas -------------- Colocar los mensajes  {\bf MSG1-}

\begin{UCtrayectoriaA}{A}{EL campo se encuentra vacío}
	\UCpaso Muestra el Mensaje \textbf{\MSGref{MSG8}{Dato requerido}}.
    \UCpaso Muestra \IUref{UIEP}{Pantalla Editar perfil} con el campo en color azul.
    \UCpaso Continua en el paso \ref{CU2Regresa} del \UCref{CU2}
\end{UCtrayectoriaA}
		
\begin{UCtrayectoriaA}{B}{El campo no tiene el formato correcto}
    \UCpaso Muestra el Mensaje \textbf{\MSGref {MSG6}{Formato incorrecto registro}}.
    \UCpaso Muestra \IUref{UIEP}{Pantalla Editar perfil} con el campo en color rojo y conservando los datos ingresados.
	\UCpaso Continua en el paso \ref{CU2Regresa} del \UCref{CU2}
\end{UCtrayectoriaA}

\begin{UCtrayectoriaA}{C}{El campo de contraseña no esta vacío}
    \UCpaso Verifica que que todos campos tengan un formato valido.\Trayref{B}
	    \\\BRref{RN2}{Datos correctos}
	\UCpaso Verifica que la nueva contraseña coincida con la contraseña de verificación.\Trayref{D}
	    \\\BRref{RN16}{Confirmación de datos}
	\UCpaso Verifica que la contraseña actual coincida con la registrada en la base de datos.\Trayref{D}\Trayref{F}
	    \\\BRref{RN18}{Cambio en la contraseña}
	\UCpaso Continua en el paso \ref{CU2Regresa2} del \UCref{CU2}
\end{UCtrayectoriaA}

\begin{UCtrayectoriaA}{D}{Los campos no coinciden}
    \UCpaso Muestra el Mensaje \textbf{\MSGref {MSG10}{Datos no coinciden}}.
    \UCpaso Se limpia el o los campos en conflicto.
    \UCpaso Muestra \IUref{UIEP}{Pantalla Editar perfil} conserva los demás datos ingresados.
	\UCpaso Continua en el paso \ref{CU2Regresa} del \UCref{CU2}
\end{UCtrayectoriaA}

\begin{UCtrayectoriaA}{E}{No se pudo actualizar la información de la cuenta}
    \UCpaso Muestra el Mensaje \textbf{\MSGref{MSG2}{Operación fallida}}.
    \UCpaso[\UCactor] Oprime el botón \IUbutton{Aceptar}.
    \UCpaso Muestra \IUref{UIEP}{Pantalla Editar perfil} con la información que se tiene almacenada en la base de datos.
    \UCpaso Continua en el paso \ref{CU2Regresa} del \UCref{CU2}
\end{UCtrayectoriaA}

\begin{UCtrayectoriaA}{F}{No hay conexión a internet}
    \UCpaso Muestra el Mensaje \textbf{\MSGref{MSG2}{Operación fallida}}.
    \UCpaso[\UCactor] Oprime el botón \IUbutton{Aceptar}.
    \UCpaso Muestra \IUref{UIEP}{Pantalla Editar perfil} conservando los datos ingresados.
    \UCpaso Continua en el paso \ref{CU2Regresa} del \UCref{CU2}
\end{UCtrayectoriaA}

\begin{UCtrayectoriaA}{R}{Se cancela el registro}
	\UCpaso[\UCactor] Oprime el botón \IUbutton{Cancelar}.
	\UCpaso Muestra el Mensaje \textbf{\MSGref{MSG4}{Confirmación}}.
    \UCpaso[\UCactor] Oprime el botón \IUbutton{Aceptar}. \Trayref{R2}
	\UCpaso Muestra la pantalla \\\IUref{UIPPrincipal}{Pantalla principal del sistema}.
	\UCpaso Termina el caso de uso.
\end{UCtrayectoriaA}

\begin{UCtrayectoriaA}{R2}{Se cancela la operación cancelar}
    \UCpaso[\UCactor] Oprime el botón \IUbutton{Cancelar} del mensaje
        \\\textbf{\MSGref{MSG4}{Confirmación}}.
    \UCpaso Muestra \IUref{UIEP}{Pantalla Editar perfil} conservando los datos ingresados.
    \UCpaso Continua en el paso \ref{CU2Regresa} del \UCref{CU2}
\end{UCtrayectoriaA}
		
%-------------------------------------- TERMINA descripción del caso de uso.