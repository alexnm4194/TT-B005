%Editar cuestionario
\begin{UseCase}{CU8}{Modificar cuestionario}{
    El psicólogo podrá modificar o eliminar los cuestionarios creados.
	}
	\UCitem{Versión}{1.0 - 10/04/2017}
	\UCitem{Autor}{Alejandro Noriega Montalban}
	\UCitem{Prioridad}{Baja}
	\UCitem{Actor}{Psicólogo}
	\UCitem{Propósito}{Modificar un cuestionario.}
	\UCitem{Entradas}{}
	\UCitem{Salidas}{}
	\UCitem{Precondiciones}{
        \begin{itemize}
            \item El cuestionario debe existir en el sistema.
        \end{itemize}
	}
	\UCitem{Postcondiciones}{
        \begin{itemize}
            \item El cuestionario quedará actualizado en el sistema.
        \end{itemize}
	}
	\UCitem{Reglas del negocio}{
	    \begin{itemize}
	        \item \BRref{RN25}{ Formato Nombres}
            \item \BRref{RN27}{ Tipos de preguntas}
            \item \BRref{RN28}{ Ignorar campos}
	    \end{itemize}
	}	
	\UCitem{Mensajes}{
    	\begin{itemize}
            \item \MSGref{MSG1}{Operación exitosa}
            \item \MSGref{MSG2}{Operación fallida}
            \item \MSGref{MSG4}{Confirmación}
            \item \MSGref{MSG6}{Formato incorrecto registro}
        \end{itemize}
    }
\end{UseCase}
% ------------ Trayectoria principal, poner referencias a pantallas del sistema, reglas entre otras cosas que sean necesarias. ------------
\begin{UCtrayectoria}{Principal}
    \UCpaso Despliega en pantalla el cuestionario  \label{CU8Continue}
        \\\IUref{UICS}{Pantalla Cuestionario seleccionado}.
    \UCpaso[\UCactor] Oprime el botón \IUbutton{Editar}.
    \UCpaso[\UCactor] Se muestran 2 preguntas con la información almacenada y el nombre actual vía.
        \\\IUref{UIN}{Pantalla Nuevo Cuestionario}.
    \UCpaso Se editan los campos deseados en la pantalla. \label{CU8Regresa2} \Trayref{D} \Trayref{D2} \Trayref{D3}
    \UCpaso[\UCactor] Oprime el botón \IUbutton{Guardar}. \Trayref{C} \Trayref{R} \Trayref{F} \label{CU8Regresa}
    \UCpaso Verifica el campo de nombre. \Trayref{A}
        \\\BRref{RN25}{ Formato Nombres}
    \UCpaso Ignora los campos vacios.
        \\\BRref{RN28}{ Ignorar campos}
    \UCpaso Muestra el Mensaje \textbf{\MSGref{MSG4}{Confirmación}}.
    \UCpaso[\UCactor] Oprime el botón \IUbutton{Aceptar}.
    \UCpaso Se guarda el cuestionario. \Trayref{B}
    \UCpaso Muestra el Mensaje \textbf{\MSGref{MSG1}{Operación exitosa}}.
    \UCpaso[\UCactor] Oprime el botón \IUbutton{Aceptar}.
    \UCpaso Muestra la pantalla \IUref{UIC}{Pantalla Cuestionarios}.
    \UCpaso Continúa en el \UCref{CU23}.
\end{UCtrayectoria}

% ---------------- Trayectorias alternativas -------------- Colocar los mensajes  {\bf MSG1-}

\begin{UCtrayectoriaA}{A}{El campo no tiene el formato correcto}
    \UCpaso Muestra el Mensaje \textbf{\MSGref {MSG6}{Formato incorrecto registro}}.
    \UCpaso Muestra \IUref{UIN}{Pantalla Nuevo Cuestionario} con el campo en color rojo, conservando los datos y el texto ingresado.
    \UCpaso Continua en el paso \ref{CU8Regresa} del \UCref{CU8}
\end{UCtrayectoriaA}

\begin{UCtrayectoriaA}{B}{No se pudo crear el cuestionario}
    \UCpaso Muestra el Mensaje \textbf{\MSGref{MSG2}{Operación fallida}}.
    \UCpaso[\UCactor] Oprime el botón \IUbutton{Aceptar}.
    \UCpaso Muestra \IUref{UIN}{Pantalla Nuevo Cuestionario} conservando los datos ingresados.
    \UCpaso Continua en el paso \ref{CU8Regresa} del \UCref{CU8}
\end{UCtrayectoriaA}

\begin{UCtrayectoriaA}{C}{Se desea eliminar el cuestionario}
    \UCpaso[\UCactor] Oprime el botón \IUbutton{Eliminar}.
    \UCpaso Muestra el Mensaje \textbf{\MSGref{MSG4}{Confirmación}}.
    \UCpaso[\UCactor] Oprime el botón \IUbutton{Aceptar}. \Trayref{R2}
    \UCpaso Se elimina el cuestionario.
    \UCpaso Muestra \IUref{UIC}{Pantalla Cuestionarios}.
    \UCpaso Termina el caso de uso.
\end{UCtrayectoriaA}

\begin{UCtrayectoriaA}{D}{Se cambia la pregunta a tipo cerrada}
    \UCpaso[\UCactor] Selecciona el check \IUbutton{carrada} para la pregunta.
    \UCpaso Se habilitan los campos para ingresar respuestas.
        \\\BRref{RN27}{ Tipos de preguntas}
    \UCpaso Se ingresa el texto en los campos de respuestas.
    \UCpaso Continua en el paso \ref{CU8Regresa2} del \UCref{CU8}
\end{UCtrayectoriaA}

\begin{UCtrayectoriaA}{D2}{Se cambia la pregunta a tipo Multiple}
    \UCpaso[\UCactor] Selecciona el check \IUbutton{Multiple} para la pregunta.
    \UCpaso Se habilitan los campos para ingresar respuestas.
        \\\BRref{RN27}{ Tipos de preguntas}
    \UCpaso Se ingresa el texto en los campos de respuestas.
    \UCpaso Continua en el paso \ref{CU8Regresa2} del \UCref{CU8}
\end{UCtrayectoriaA}

\begin{UCtrayectoriaA}{D3}{Se cambia la pregunta a tipo libre}
    \UCpaso[\UCactor] Selecciona el check \IUbutton{Libre} para la pregunta.
    \UCpaso Se limpian los campos de respuestas.
        \\\BRref{RN27}{ Tipos de preguntas}
    \UCpaso Continua en el paso \ref{CU8Regresa2} del \UCref{CU8}
\end{UCtrayectoriaA}

\begin{UCtrayectoriaA}{F}{Siguientes preguntas}
    \UCpaso[\UCactor] Oprime el botón \IUbutton{Ver más}.
    \UCpaso Muestra \IUref{UIN}{Pantalla Nuevo Cuestionario} con el campo del nombre y los datos almacenados para en las preguntas.
    \UCpaso Continua en el paso \ref{CU8Regresa2} del \UCref{CU8}
\end{UCtrayectoriaA}

\begin{UCtrayectoriaA}{R}{Se cancela el registro}
    \UCpaso[\UCactor] Oprime el botón \IUbutton{Cancelar}.
    \UCpaso Muestra el Mensaje \textbf{\MSGref{MSG4}{Confirmación}}.
    \UCpaso[\UCactor] Oprime el botón \IUbutton{Aceptar}. \Trayref{R2}
    \UCpaso Muestra \IUref{UIC}{Pantalla Cuestionarios}.
    \UCpaso Termina el caso de uso.
\end{UCtrayectoriaA}

\begin{UCtrayectoriaA}{R2}{Se cancela la operación anterior}
    \UCpaso[\UCactor] Oprime el botón \IUbutton{Cancelar} del mensaje
        \\\textbf{\MSGref{MSG4}{Confirmación}}.
    \UCpaso Muestra \IUref{UIN}{Pantalla Nuevo Cuestionario} conservando los datos ingresados.
    \UCpaso Continua en el paso \ref{CU8Regresa} del \UCref{CU8}
\end{UCtrayectoriaA}
%-------------------------------------- TERMINA descripción del caso de uso.