%Buscar paciente
\begin{UseCase}{CU5}{Visualizar perfil del paciente}{
    El psicólogo accesedará al menú de acciones para cada paciente y a la información del mismo.
	}
	\UCitem{Versión}{1.0 - 12/04/2017}
	\UCitem{Autor}{Alejandro Noriega Montalban}
	\UCitem{Prioridad}{Alta}
	\UCitem{Actor}{Psicólogo}
	\UCitem{Propósito}{El psicólogo vea la información del paciente.}
	\UCitem{Entradas}{}
	\UCitem{Salidas}{}
	\UCitem{Precondiciones}{
        \begin{itemize}
            \item El expediente del paciente debe existir en el sistema.
        \end{itemize}
	}
	\UCitem{Postcondiciones}{
	}
	\UCitem{Reglas del negocio}{
	}	
	\UCitem{Mensajes}{
	}
\end{UseCase}
%.... Pendiente reglas del neogocio

% ------------ Trayectoria principal, poner referencias a pantallas del sistema, reglas entre otras cosas que sean necesarias. ------------
\begin{UCtrayectoria}{Principal}
    \UCpaso Muestra la pantalla \\\IUref{UIPPrincipal}{Pantalla principal del sistema}.
    \UCpaso[\UCactor] Oprime el botón \IUbutton{Ver Pacientes}.
    \UCpaso Muestra \IUref{UIVP}{Pantalla Ver pacientes}. \Trayref{R}
    \UCpaso[\UCactor] Selecciona al paciente.
    \UCpaso Muestra la pantalla \\\IUref{UIP}{Pantalla Información paciente}.
\end{UCtrayectoria}

% ---------------- Trayectorias alternativas -------------- Colocar los mensajes  {\bf MSG1-}
\begin{UCtrayectoriaA}{R}{Se oprime regresar}
	\UCpaso[\UCactor] Oprime el botón \IUbutton{Regresar}.
	\UCpaso Muestra la pantalla \\\IUref{UIPPrincipal}{Pantalla principal del sistema}.
	\UCpaso Termina el caso de uso.
\end{UCtrayectoriaA}
%-------------------------------------- TERMINA descripción del caso de uso.