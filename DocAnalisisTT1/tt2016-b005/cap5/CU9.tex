%visualizar cuestionario
\begin{UseCase}{CU9}{Visualizar Cuestionario}{
    El psicólogo podrá ver el cuestionario seleccionado.
	}
	\UCitem{Versión}{1.0 - 11/04/2017}
	\UCitem{Autor}{Alejandro Noriega Montalban}
	\UCitem{Prioridad}{Alta}
	\UCitem{Actor}{Psicólogo}
	\UCitem{Propósito}{Visualizar un cuestionario.}
	\UCitem{Entradas}{}
	\UCitem{Salidas}{}
	\UCitem{Precondiciones}{
        \begin{itemize}
            \item El cuestionario debe existir en el sistema.
        \end{itemize}
	}
	\UCitem{Postcondiciones}{
	}
	\UCitem{Reglas del negocio}{
	}	
	\UCitem{Mensajes}{
    }
\end{UseCase}
%.... Pendiente reglas del neogocio

% ------------ Trayectoria principal, poner referencias a pantallas del sistema, reglas entre otras cosas que sean necesarias. ------------
\begin{UCtrayectoria}{Principal}
    \UCpaso Muestra la pantalla \IUref{UIPPrincipal}{Pantalla principal del sistema}.
    \UCpaso[\UCactor] Oprime el botón \IUbutton{Cuestionarios}
    \UCpaso Muestra \IUref{UIC}{Pantalla Cuestionarios}.
    \UCpaso[\UCactor] Selecciona un cuestionario.
    \UCpaso Despliega en pantalla el cuestionario vía
        \\\IUref{UICS}{Pantalla Cuestionario seleccionado}.
    \UCpaso[\UCactor] Oprime el botón \IUbutton{Cerrar}.
    \UCpaso Muestra \IUref{UIC}{Pantalla Cuestionarios}.
    \UCpaso Continúa paso \ref{CU8Continue} en el \UCref{CU8}.
\end{UCtrayectoria}

% ---------------- Trayectorias alternativas -------------- Colocar los mensajes  {\bf MSG1-}

%-------------------------------------- TERMINA descripción del caso de uso.