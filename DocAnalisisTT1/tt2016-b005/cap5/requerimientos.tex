\section{Requerimientos}

\subsection{Requerimientos funcionales}

\noindent
ID: RF-SAP1\\
Nombre: Registro Psicólogo\\
Descripción: El psicólogo debe capturar los siguientes datos en el sistema, para crear su usuario
dentro de este. Se validarán los datos para que sean correctos y no exista algún otro psicólogo con la misma cédula profesional y/o usuario.

\begin{itemize}
    \item Nombre
    \item Primer Apellido
    \item Segundo Apellido *
    \item Cédula profesional
    \item Fecha de nacimiento
    \item Correo electrónico
    \item Confirmar correo electrónico
    \item Teléfono *
    \item Teléfono celular *
    \item Contraseña
    \item Confirmar contraseña
    \item Nota: * campos opcionales.
\end{itemize}
\vspace{4cm}
\\

\noindent
ID: RF-SAP2\\
Nombre: Registro Pacientes\\
Descripción: El psicólogo debe capturar los siguientes datos en el sistema, para crear el expediente del paciente dentro de este. Se validarán los datos para que sean correctos.

\begin{itemize}
    \item Nombre
    \item Primer Apellido
    \item Segundo Apellido *
    \item Fecha de nacimiento
    \item Boleta institucional IPN
    \item Correo electrónico
    \item Teléfono *
    \item Teléfono celular *
    \item Calle
    \item No. de casa
    \item Colonia
    \item Código postal
    \item Municipio o Delegación
    \item Estado
    \item Enfermedades*
    \item Medicamentos*
    \item Nota: * campos opcionales.
\end{itemize}
\\

\noindent
ID: RF-SAP3\\
Nombre: Generar cuestionarios\\
Descripción: El psicólogo deberá poder generar cuestionarios de forma dinámica para ser aplicados a los pacientes cuando así se le consideré conveniente.
\\

\noindent
ID: RF-SAP4\\
Nombre: Generar reportes\\
Descripción: El sistema deberá generar reportes sobre el uso y cantidad de pruebas que se han realizado en un periodo de tiempo, al igual que el numeró de pacientes relacionado a la aplicación de las pruebas.
\\

\noindent
ID: RF-SAP5\\
Nombre: Desplegar módulos de la prueba HTP\\
Descripción: El sistema deberá desplegar un menú el cual contenga los tres módulos que componen la prueba HTP para que el psicólogo seleccione el que realizará el paciente.
\\

\noindent
ID: RF-SAP6\\
Nombre: Analizar dibujo de la casa de HTP\\
Descripción: Se deberán encontrar patrones y elementos dentro del esquema que correspondan a la imagen de una casa para que con base en ellos se busquen coincidencias con los rasgos descritos por el autor de la prueba HTP. 
\\

\noindent
ID: RF-SAP7\\
Nombre: Analizar dibujo del árbol de HTP\\
Descripción: Se deberán encontrar patrones y elementos dentro del esquema que correspondan a la imagen de un árbol para que con base en ellos se busquen coincidencias con los rasgos descritos por el autor de la prueba HTP. 
\\

\noindent
ID: RF-SAP8\\
Nombre: Analizar dibujo de la persona de HTP\\
Descripción: Se deberán encontrar patrones y elementos dentro del esquema que correspondan a la imagen de una persona para que con base en ellos se busquen coincidencias con los rasgos descritos por el autor de la prueba HTP. 
\\

\noindent
ID: RF-SAP9\\
Nombre: Capturar anotaciones (observaciones)\\
Descripción: Dentro del sistema el psicólogo deberá tener un espacio para poder tomar notas, escribir párrafos en contexto libre que el consideré oportunos.
\\

\noindent
ID: RF-SAP10\\
Nombre: Administrar pacientes\\
Descripción: El registro de los pacientes deberá poder ser modificado en caso que existiera algún error en su captura y/o faltará algún dato por capturar.
\\

\noindent
ID: RF-SAP11\\
Nombre: Administrar perfil del psicólogo\\
Descripción: El psicólogo podrá modificar la información de su perfil.
\\

\noindent
ID: RF-SAP12\\
Nombre: Controlar acceso\\
Descripción: El  sistemas  solo  deberá  permitir  el  acceso a un psicólogo a la vez,  a  los  expedientes  de  cada uno de sus paciente, de igual forma la información de cada expediente deberá estar protegida por una contraseña.
\\
\newpage
\subsection{Requerimientos no funcionales}

\begin{NFRequieriments}
\NFRitem{1}{Mantenibilidad}{
\begin{itemize}
\item Todo nuevo requerimiento funcional o no funcional debe ser analizado para poder apreciar el impacto que éste tendrá sobre el sistema y proceso de negocio.
\end{itemize}
}

\NFRitem{2}{Eficiencia}{
\begin{itemize}
\item Toda funcionalidad local del sistema y transacción de negocio debe responder al usuario en menos de 10 segundos.
\item El sistema seguirá trabajando aun sin conexión a internet, en cuanto se restablezca la conexión se sincronizarán las base de datos local con la base de datos en la nube.
\end{itemize}
}

\NFRitem{3}{Seguridad}{
\begin{itemize}
\item Los datos personales de los pacientes estarán bajo un cifrado AES.
\item La contraseña del psicólogo estará bajo un cifrado MD5, aplicado tres veces al dato.
\item Los formularios para ingresar datos estarán validados por tipo de dato y longitud.
\end{itemize}
}

\NFRitem{4}{Usabilidad}{
\begin{itemize}
\item El tiempo de aprendizaje del sistema por un usuario deberá ser menor a dos semanas.
\item El sistema debe proporcionar mensajes de error que sean informativos y orientados a usuario final.
\item El sistema debe contar con una pantalla para contacto con los desarrolladores para cualquier tipo de aclaración.
\item La aplicación debe de garantizar la adecuada visualización en tabletas con una dimensión en pantalla de 12.9 pulgadas.
\item El sistema debe poseer interfaces gráficas bien formadas.
\end{itemize}
}

\NFRitem{5}{Extensibilidad}{
\begin{itemize}
\item El sistema podrá tener un crecimiento futuro ya que estará programado por módulos para permitir la integración de nuevas pruebas.
\end{itemize}
}

\end{NFRequieriments}