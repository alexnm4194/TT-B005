\section{Contexto}

El campo de la psicología es uno de los menos considerados para la implementación de sistemas, esto debido a la naturaleza de la psicología que es subjetiva y depende de numerosas variables que no siempre pueden ser computables ya que no es una ciencia exacta, esto provoca que no se puedan desarrollar modelos computacionales que entreguen resultados precisos, ya que no solo dependen del sujeto de estudio, involucran la formación, experiencia y criterio del analista.

El uso de herramientas psicológicas como pruebas proyectivas gráficas, permite al área de sistemas computacionales tomar los elementos cuantitativos y cualitativos que pueden ser computados como lo es el trazo, la posición y la presión ejercida sobre dispositivos tecnológicos, y de esta manera poder diseñar prototipos de modelos computacionales que faciliten el trabajo de los psicólogos y de esta manera impulsar y promover el uso de tecnologías en esta área del conocimiento.