\section{Justificación}

\subsection{Problema}
Se ha identificado que hoy en día en el departamento de atención especializada del \textit{Centro de Estudios Científicos y Tecnológicos No. 1 ``Gonzalo Vázquez Vela'' del Instituto Politécnico Nacional} el manejo de expedientes es tedioso, requiere tiempo y espacio el mantener organizado el material de forma física, en las situaciones donde se aplica la prueba HTP este forma parte del expediente y con el paso del tiempo estas pruebas pierden fidelidad al deteriorarse la hoja y el dibujo, esto provoca que no se tenga un historial confiable donde pueda observar la evolución del paciente en las distintas iteraciones que hace de la prueba. Por las características de esta, se presentan varios problemas en la interpretación, ya que depende mucho la formación y experiencia del psicólogo aplicando dicha prueba.

\subsection{Propuesta}
Ante Dicha situación se propone, diseñar y desarrollar un sistema de apoyo a psicólogos enfocado a realizar la prueba HTP, los tres elementos que conforman HTP se dividirán en módulos diferentes para permitir el análisis individual, utilizando un dispositivo móvil con tecnología táctil y capacidad sensitiva a la presión, en el cual el psicólogo pueda aplicar dicha prueba de forma digital, para obtener un perfil previo y no concluyente, este perfil formara parte del expediente digital del paciente y brindara parámetros con un grado de objetividad formando una base que facilite la interpretación y detección de conflictos o desajustes emocionales sin descuidar la interacción interpersonal que ayuden al psicólogo a dar un diagnóstico de la personalidad del individuo.

Este sistema permitirá configurar el orden de aplicación de los elementos de la prueba, estando dentro de las posibilidades una casa, un árbol, o una persona para poder definir el orden de los patrones que se buscaran, y así obtener los identificadores necesarios para realizar el análisis, en conjunto a esto se obtendrá la cantidad de presión que hace el paciente sobre el dispositivo móvil, ya que este es otro indicador de personalidad que sirve para un mejor resultado.

El sistema es planteado para su uso en un dispositivo móvil con el fin de permitir al psicólogo poder aplicar la prueba en entornos diferentes sin necesidad de ubicarse en su área de trabajo fija, permitiéndole llevar con él los expedientes de los diferentes pacientes sin la dificultad que conlleva transportar expedientes físicos.

Para poder realizar este sistema de apoyo a psicólogos es necesario la aplicación de conocimientos adquiridos durante la carrera de Ingeniería en Sistemas Computacionales, en particular lo visto en materias como Desarrollo de aplicaciones para dispositivos móviles, ya que se utilizarán dispositivos móviles, Base de datos puesto que se necesita gestionar la información de los pacientes y psicólogos, así como los datos necesarios para analizar y evaluar cada elemento de la prueba, Reconocimiento de patrones para poder reconocer distintas partes de los trazos de las figuras solicitadas en la prueba HTP.