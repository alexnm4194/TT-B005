\section{Prueba Proyectiva HTP}

HTP (casa, árbol, persona)
\\\\
La prueba de HTP (son las siglas en inglés de Casa, Árbol, Persona), cuyo creador fue John Buck (psicólogo estadounidense), quien lo creó en el año 1948. Consiste en pedirle al sujeto que dibuje en hojas en blanco cada uno de estos elementos; este tipo de test permite al sujeto proyectar con más facilidad sus áreas de conflicto y elementos de su personalidad, es decir, se establece un tipo de comunicación eficaz en la que elementos conscientes e inconscientes salen a relucir en un ambiente terapéutico.

Concretamente, a través de estos dibujos se podrá ver cómo un individuo experimenta su yo en relación con los demás y el entorno familiar.

Este test está compuesto de dos fases. La primera fase podemos definirla como no verbal y creativa, es el momento en el que el sujeto dibuja una casa, un árbol o una persona en función de la consigna que se le haya dado. En la segunda fase, el sujeto cuenta la historia de cada uno de estos elementos en los tres tiempos (pasado, presente o futuro), también hay psicólogos que realizan una serie de preguntas estructuradas. [ref]
\\\\\\\\\\