\section{¿Qué es la psicologia?}

La psicología es la ciencia de la conducta humana, o sea de las acciones.
Desde hace mucho tiempo se ha considerado a la psicología como la ciencia que estudia
los fenómenos que acontecen en el interior del individuo, es decir, en la intimidad de la
conciencia. Este criterio cierto, sin embargo no basta, y en la actualidad es sustituido
por otro. El individuo está sujeto a la influencia del medio al que tiene que adaptarse.
Así, por ejemplo, el aumento de la temperatura en un clima cálido acelera los
fenómenos fisiológicos tales como la circulación y la respiración. El corazón late mayor
número de veces y los cambios respiratorios se intensifican. El hombre, en vista de ello,
ajusta su modo de ser según las circunstancias y procura suavizar la acción externa:
construye sus habitaciones apropiadas para el caso, dispone sus vestidos y norma su
alimentación. Todo esto origina una conducta que es el punto de partida de una serie de
conductas derivadas. Los seres al obrar no solamente lo hacen con una finalidad
externa, sino también interna, y en esto hay que diferenciar lo fisiológico de lo
psicológico; lo corporal de lo mental. [ref]