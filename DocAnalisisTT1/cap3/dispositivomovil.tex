\section{¿Qué es un dispositivo móvil?}

Un dispositivo móvil se puede definir como un aparato de pequeño tamaño, con algunas capacidades de procesamiento, con conexión permanente o intermitente a una red, con memoria limitada, que ha sido diseñado específicamente para una función, pero que puede llevar a cabo otras funciones más generales. De acuerdo con esta definición existen multitud de dispositivos móviles, desde los reproductores de audio portátiles hasta los navegadores GPS, pasando por los teléfonos móviles, los PDAs o los Tablet PCs. [ref]

\subsection{Tableta}

Una computadora de propósito general contenida en un panel de pantalla táctil. Aunque las computadoras de tableta anteriores requieren un lápiz, tabletas modernas son operadas por los dedos, o un lápiz óptico opcional [ref]. 